%! Author = florian
%! Date = 6/23/23
\subsection{ErrorProne}\label{subsec:errorprone}
Error Prone is a popular Java library that aims to improve the quality and reliability of Java code by detecting and preventing common programming errors.
It functions as a static analysis tool and compiler plugin, seamlessly integrating with the Java compiler to provide valuable feedback and suggestions during the development process.

Issues are flagged by creating compiler errors with an appropriate message, which contextualizes errors and often include a possible fix.
Similar to how modern IDEs often suggest a possible fix, before even compiling the source code.
Error Prone is also able to create patch files with these fixes, that then can be applied by the developers.

Error Prone can be extended by writing plugins.
The plugins operate directly on the abstract syntax tree, which helps writing powerful plugins.
The downside of this approach is that writing plugins is not an easy task.

Refaster is a tool that aims to solve this problem.
It provides an API with before and after templates.
That way it is easy to create customized plugins for Error Prone.

A similar tool to Error Prone is Spotbugs, which is described in the Subsection\ \ref{subsec:spotbugs}.
In comparison to Spotbugs, Error Prone offers a more powerful configuration and plugin system, making Error Prone the preferred static analysis tool~\cite{errorprone}.

\subsubsection{Test Setup}
Error Prone is easy to integrate in existing gradle projects.
Similar to Checkstyle, which is described in Subsection\ \ref{subsec:checkstyle}.
The plugin simply has to be included, and it will integrate itself into the \texttt{compileJava}-task.
The Figure\ \ref{fig:errorprone-gradle} shows the configuration in the \texttt{build.gradle}.

\begin{figure}[h]
    \caption{Checkstyle build.gradle}
    \lstinputlisting[basicstyle=\tiny,label={lst:errorprone-gradle}]{../../errorprone/build.gradle}
    \label{fig:errorprone-gradle}
\end{figure}

\subsubsection{Nullaway}
A plugin which should be mentioned is Nullaway.
It helps developers finding possible NullPointerExceptions in the code, which is still a problem in Java, since there is no real option type, like in \.Net or TypeScript.
By using Nullaway developers can mitigate the risk of unexpected NullPointerExceptions.

\subsubsection{Comparison to SonarQube}
Error Prone and Sonar Qube are both static analysis tools, which means they solve the same problems.
Both have standard bug patterns to find bugs and both can be extended to add functionality.

As seen above, it is easy to add Error Prone to the build process, and it integrates itself directly into the compile task.
That way it is close to the development cycle, which gives more feedback to the developers.

Both Error Prone and SonarQube offer a plugin system.
Error Prones offers a number of well known plugins, such as Nullaway, which are often used extensively.

One downside of Error Prone is, that it is limited to the JVM.

In the end, they still can be combined, to offer the biggest coverage of bug patterns possible.
