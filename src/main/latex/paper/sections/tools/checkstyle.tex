\subsection{Checkstyle}\label{subsec:checkstyle}
Checkstyle is a static code analysis tool for Java that checks code against a configurable coding standard.
It mainly focuses on coding style and formatting issues, such as indentation, naming conventions, and Javadoc comments.\cite{checkstyle-docs}
Some key aspects of Checkstyle include:

\begin{itemize}
    \item\textbf{Highly configurable}: Checkstyle allows you to define your own coding standards and rules.
    \item\textbf{Simple and lightweight}: It is easy to integrate into build systems like Maven and Ant, as well as IDEs like IntelliJ and Eclipse.
    \item\textbf{Limited to Java}: Checkstyle is specifically designed for Java and does not support other programming languages.
\end{itemize}

\subsubsection{Test Setup}
Setting up checkstyle is rather easy.
It only has to be included into the build script.
In this case gradle with the groovy DSL was used as seen in figure\ \ref{fig:checkstyle-gradle}~\cite{gradle-checkstyle}.

\begin{figure}[h]
    \caption{Checkstyle build.gradle}
    \lstinputlisting[basicstyle=\tiny,label={lst:checkstyle-gradle}]{../../checkstyle/build.gradle}
    \label{fig:checkstyle-gradle}
\end{figure}

This automatically adds the checkstyle task, that can be executed by running \texttt{gradle checkstyleMain}.

\subsubsection{Comparison to SonarQube}
The main difference between Checkstyle and SonarQube is that it Checkstyle focuses on finding misformatted code, while SonarQube only finds code smells.
This also means one can configure Checkstyle in a way that it forces formatting, which SonarQube will identify as a code smell.

Checkstyle is not just a replacement for some SonarQube features, but even complements them, since they intend to solve different problems.
Checkstyle is a tool that only focuses on formatting issues, this is not a feature that SonarQube inherently implements.

SonarQube implements their own rules, which are called code smells.
Code smells sometimes overlap with formatting issues, but this does not have to be the case.
Code smells are indicators that show deeper problems with the code, for example god classes or procedural programming in an OOP language.
For example duplicated code, long methods or classes, and excessive method parameters are code smells.
Important to note is that code smells are not necessary bugs or semantic errors.
Code with many smells can still run perfectly fine, but it might be harder to read or maintain.

This makes formatting and code smells two distinct problems, that sometimes can overlap.
