\subsection{SpotBugs}\label{subsec:spotbugs}

SpotBugs is a static analysis tool that checks Java code for common bug patterns, such as missing null checks and deadlocks.
It categorizes bugs into different types, such as, but not limited to, correctness, bad practice, performance, style, experimental and security.
Which allows for different levels of severity to be assigned to each bug type ~\cite{spotbugs-docs}.

SpotBugs is the successor of the now abandoned project FindBugs, that was developed by the University of Maryland ~\cite{spotbugs-docs}.

\subsubsection{Setup}

Similar to Checkstyle as seen in section ~\ref{subsubsec:checkstyle-test-setup}, SpotBugs can be easily integrated into a Gradle build script.
This is shown by figure ~\ref{fig:spotbugs-gradle}, which also uses the groovy DSL.

\begin{figure}[h]
    \caption{SpotBugs build.gradle}
    \lstinputlisting[basicstyle=\tiny,label={lst:spotbugs-gradle}]{../../spotbugs/build.gradle}
    \label{fig:spotbugs-gradle}
\end{figure}

This automatically adds the spotbugsMain task, that can be executed by running \texttt{gradle spotbugsMain}.

\subsubsection{Comparison to SonarQube}\label{subsubsec:spotbugs-comparison}

While SpotBugs performs a similar task to SonarQubes static analysis, it can only be used for Java code.

It is also hard to find a direct comparison on the bugs they can detect, since SonarQube does not provide a list of all the rules it uses,
but knowing it supports multiple languages, it is reasonable to assume that it can detect more bugs than SpotBugs.

While setting up SpotBugs is simpler than SonarQube, setting up SonarQube is still not a difficult task and can be done in a few minutes.
SpotBugs also does not offer many configuration options, while SonarQube allows for a lot of customization and offers a lot more in general.

So if one is only looking for a simple tool to check for bugs in Java code, SpotBugs is a good choice, however if one is looking for a tool that can be used for multiple languages and offers more features, SonarQube would be the better choice.