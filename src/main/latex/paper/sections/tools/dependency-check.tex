\subsection{OWASP Dependency Check}\label{subsec:owasp-dependency-check}

OWASP Dependency Check is a tool for finding known vulnerabilities in project dependencies.
It is developed by OWASP, the Open Worldwide Application Security Project,
a non-profit online community that produces freely-available knowledge like articles,
methodologies, documentation, tools, and technologies in the field of web application security ~\cite{owasp-about}.

Dependency check utilizes known vulnerabilities to find vulnerable dependencies.
These known vulnerabilities are sourced from an open database called the National Vulnerability Database (NVD) that is maintained
by the National Institute of Standards and Technology (NIST) ~\cite{nvd}.

Dependency Check supports multiple programming languages and package managers through the use of \emph{analyzers}.
When an analyzer is run on a project it will scan the project for dependencies and then compare them to the NVD database.
If a dependency is found that is vulnerable,
evidence of the vulnerability is collected (vendor, product and version) and a report is generated ~\cite{owasp-dependency-check-work}.

\subsubsection{Usage}

Dependency Check can be used as a command line tool or as a plugin for build tools like Gradle and Maven.
The command line tool can be used for all supported programming languages and package managers, while the plugins are limited to the build tool they are made for.

Using the command line tool is as simple as running \texttt{\scriptsize dependency-check.sh --project "My App Name" --scan "java/application/lib"}, where \texttt{\scriptsize--project} is the name of the project and \texttt{\scriptsize--scan} is the path to the files that will be scanned.

There is also a Jenkins plugin available that can be used to integrate Dependency Check into a Jenkins Pipeline,
automatically generating a report and failing the build if a vulnerability is found.

\subsubsection{Conclusion}\label{subsubsec:depencency-check-conclusion}

While it is simple to use and certainly a useful tool,
most modern build tools and IDEs already have a built-in dependency checker that reports vulnerabilities.
This makes it redundant to use Dependency Check in most cases, though it is a different story if dependency checking should be integrated into a CI/CD pipeline.
In this case, Dependency Check is a good alternative to SonarQube's dependency checking.

