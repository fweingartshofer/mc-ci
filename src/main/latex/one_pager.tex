%! Author = florian
%! Date = 3/15/23

% Preamble
\documentclass[11pt,journal]{IEEEtran}

% Packages
\usepackage{amsmath}
\usepackage[english]{babel}
\usepackage{url}
\usepackage{listings}
\usepackage{libertine}
\usepackage{libertinust1math}
\usepackage[T1]{fontenc}
\usepackage{hyperref}
\usepackage[a4paper, total={6in, 9in}]{geometry}
\usepackage{graphicx}

\title{Making CI Open Source}
\author{
    Haas, Fabian\\
    \and
    Weingartshofer, Florian
}
\date{\today}

% Document
\begin{document}
    \maketitle


    \section{Introduction}\label{sec:introduction}
    The aim of this paper is to find alternatives to SonarSources SonarCloud product.
    SonarQube is open source and SonarCloud is based on that, there are many limiting factors for SonarQube, such as not having the possibility to use it in pull requests.
    So only the core is open source, and it is hard to integrate it into existing CI/CD pipelines, such as GitHub Actions.

    \section{Motivation}\label{sec:motivation}
    As described in\ \ref{sec:introduction} SonarQube only offers limited functionality in its open source variant.
    SonarSource offers a free tier for open source projects, but not for private git repositories.
    This limits organizations that are not ready to publish their source code on a git host.
    These organizations would have to pay for the paid tier, even when they intend to publish their source code later on.

    \section{Tools to Investigate}\label{sec:tools-to-investigate}
    All tools that are investigated should be open source and free to use.
    They should replace specific SonarQube functions, such as code style or security vulnerability scans.
    The following tools will be investigated and evaluated.
    \begin{itemize}
        \item Checkstyle: A tool to check if code conforms to a common style guide
        \item SpotBugs: A static analysis tool that checks code for common bug patterns, such as missing null checks
        \item ErrorProne: Another static analysis tool, which also checks for common bug patterns
        \item Dependency Check: Dependency vulnerability scanner endorsed by OWASP
        \item Other tools could be Jenkins CI Build Server or GitHub DependaBot
    \end{itemize}

    \section{Result}\label{sec:result}
    The result should be a guide on how to use these tools in a CI environment.
    Furthermore, should the tools be evaluated and their strengths and shortcomings should be briefly described.

    \section{Project Team}\label{sec:project-team}
    \begin{itemize}
        \item Fabian Haas
        \item Florian Weingartshofer
    \end{itemize}
\end{document}